\documentclass[review]{elsarticle}

\usepackage{lineno,hyperref}
\modulolinenumbers[5]

\journal{Journal of \LaTeX\ Templates}
\usepackage{amssymb}
\usepackage{amsmath,amsthm}
\usepackage{float}
\usepackage{url}
\usepackage{setspace}
\usepackage{hyperref}
\usepackage{todonotes}
\usepackage{amsthm} 
\usepackage[ruled,vlined,linesnumbered]{algorithm2e} 
\usepackage{makecell}

\onehalfspacing
\usepackage{pdfpages}
\usepackage{times}
\usepackage{multirow}
\usepackage[toc,page]{appendix}
\usepackage{listings}
\newtheorem{lemma}{Lemma}
\newtheorem{definition}{Definition}
\newtheorem{theorem}{Theorem}
\newtheorem{claim}{Claim}
\newtheorem{example}{Example}
\newtheorem{proposition}{Proposition}


\DeclareMathOperator{\supp}{support}
\DeclareMathOperator{\support}{support}
\DeclareMathOperator{\Trim}{Trim}
\DeclareMathOperator{\LTrim}{LTrim}

\begin{document}

\begin{definition}\label{consecutive} For a set $S \subseteq \mathbb{R}$ we say that
$B \subseteq S$ is \emph{consecutive} if any $s\in S$ that is smaller than $\max B$ and larger than $\min B$ is in $B$.
\end{definition}

\begin{definition}\label{part_consecutive} A partition $P=\{B_1,\dots,B_n\}$ of a set $S \subseteq \mathbb{R}$ is called 
\emph{consecutive} if all the subsets $B_1 ,\dots,B_n$ are consecutive.
\end{definition}


\begin{definition}\label{partition}
	For a discrete real random variable $X$ and a partition $P$ of its support, we define a new discrete random variable $X_P$ by:
	
	$$Pr(X_P = t) = \begin{cases} 
	Pr(X\in B) & \text{If $t$ is the minimum of some } B \in P, \\
	0      & \text{otherwise.}
	\end{cases}$$
\end{definition}


\begin{definition}
	For discrete real-valued variables $X_1$ and $X_2$, we say that $X_1$ is an $\varepsilon$ one-sided Kolmogorov approximation of $X_2$, denoted by 
	$X_1 \preceq_\varepsilon X_2$, if $$0 \leq  \max\limits_{t} (F_{X_1}(t)-F_{X_2}(t)) \leq \varepsilon.$$
\end{definition}

\begin{proposition}
	For any discrete real-valued random variable $X$ and any $m \in \mathbb{N}$, there is a consecutive partition $P$ of $\supp(X)$ such that $|P|\leq m$ and
	$X \preceq_\varepsilon X_P$ 
	%and there is no $\varepsilon' < \varepsilon$ and $X'$ with $\supp(X')\leq m$ such that $X \preceq_{\varepsilon'} X'$.
	where $$ \varepsilon = \min\{ \varepsilon \colon \text{ $\exists X'$ with $\supp(X')\leq m$ such that $X \preceq_\varepsilon X'$}\}.$$
\end{proposition}


%\begin{proposition}
%	For any discrete real-valued random variable $X$ and any $m \in \mathbb{N}$,
%	if $$X' =\arg \min_{\stackrel{X'}{\supp(X')\leq m}} \max\limits_{t}|F_X(t)-F_{X'}(t)|$$ 
%	then there is a consecutive partition $P$ of $\supp(X)$ such that $X'= X_P$.
%\end{proposition}



%\begin{definition}
%Partition of random variable $X$ means a partition over the set $\support(X)$ where each subset represents by one of its members and forms a new random variable, $X'$. In our case, each subset represents by its minimal value.
%\end{definition}

%\begin{definition}\label{partition}
%The random variable $X'$ is a partition of the random variable $X$ if there is a partition $S$ of the set $\support(X)$ and a function $f:(B\in S) \rightarrow \mathbb{N}$ such that $\forall B\in S\rightarrow f(B)\in \support(X')$ and  $P_{X'}(f(B))=\sum_{b\in B} P_X(b)$
%\end{definition}

%\begin{definition}\label{partition_min}
%If $f:=min$. The random variable $X'$ is a partition of the random variable $X$ if there is a partition $S$ over the set $\support(X)$ such that $\forall B\in S\rightarrow min(B)\in \support(X')$ and $P_{X'}(min(B))=\sum_{b\in B} P_X(b)$
%\end{definition}

%\begin{lemma}\label{consecutive}
%Given $X$ a random variable with a support of size $n$ and $X'$ a random variable with a support of size $m$ which is the optimal partition of $X$ such that $\max\limits_{t}|F_X(t)-F_{X'}(t)|$ is minimal. Then, the partition $X'$ is consecutive.
%\end{lemma}

\begin{proof}
$X_S$ is a random variable defined by the partition $S$, we will show that $\forall B\in S$, $B$ is consecutive.
Proof by contradiction, assuming that $S$ is sorted by the subsets minimum values and $\exists B \in S $ which is not consecutive. Let us look at the first subset, following $B$ in ascending order, $C$, clearly $min(B)<min(C)$ and $min(B),min(C) \in \support(X_S)$. Notice that all the values smaller then $min(C)$ are in the subsets preceding $C$. Assuming that there is a value $y\in support(X)$ smaller than $min(C)$ and $y$ is not a member in one of the subsets preceding $C$, therefore $y$ must be a member in one of the subsets following $C$ including $C$ itself. However, if that was the case, $y$ would be the minimum value of its subset and by that $min(C)$ would not be the minimum value of the subset following $B$ in contradiction to our construction. By that we know that every value smaller than $min(C)$ is in one of the subsets preceding it. But, there may be values that are grater than $min(C)$ in one of the subsets preceding $C$ which obviously increasing $\max\limits_{t}|F_{X}(t)-F_{X_S}(t)|$, we can remove them from their subsets and add them to $C$ and by that decrease $\max\limits_{t}|F_{X}(t)-F_{X_S}(t)|$, in contradiction to the optimality of $X_P$. Adding those values to $C$ will only decrease $F_{X_S}(t), \forall t\in \support(X)<min(C)$ and remain the same for $F_{X_S}(t), \forall t \in \support\geq min(C)$.
\end{proof}

%\begin{proof}
%$X'$ is a partition of $X$ by definition  ~\ref{partition_min} means that there is a partition $S$ over the set $\support(X)$ such that $\forall B\subseteq S, min(B)\in \support(X')$ and that $P_{X'}(min(B))=\sum_{b\in B} P_X(b)$. We will show that $\forall B\subseteq S$, $B$ is consecutive.
%Proof by contradiction, assuming that $S$ is sorted by the subsets minimum values and $\exists B \subseteq S $ which is not consecutive. Let us look at the first subset, following $B$ in ascending order, $C$, clearly $min(B)<min(C)$ and $min(B),min(C) \in \support(X')$. Notice that all the values smaller then $min(C)$ are in the subsets preceding $C$. Assuming that there is a value $y\in support(X)$ smaller than $min(C)$ and $y$ is not a member in one of the subsets preceding $C$, therefore $y$ must be a member in one of the subsets following $C$ including $C$ itself. However, if that was the case, $y$ would be the minimum value of its subset and by that $min(C)$ would not be the minimum value of the subset following $B$ in contradiction to our construction. By that we know that every value smaller than $min(C)$ is in one of the subsets preceding it. But, there may be values that are grater than $min(C)$ in one of the subsets preceding $C$ which obviously increasing $\max\limits_{t}|F_{X'}(t)-F_{X}(t)|$, we can remove them from their subsets and add them to $C$ and by that decrease $\max\limits_{t}|F_{X'}(t)-F_X(t)|$, in contradiction to the optimality of $X'$. Adding those values to $C$ will only decrease $F_{X'}(t), \forall t\in \support(X)<min(C)$ and remain the same for $F_{X'}(t), \forall t \in \support\geq min(C)$.
%\end{proof}

\begin{theorem}
Given $X_1 \dots X_k$ discrete random variables in a table representation and $m \in \mathbb{N}$, if $X'=OptimalTrim(X_1 \dots X_k,m)$, a discrete random variable and a partition of $X$ with a support of size $\leq m$ then $\max\limits_{t}|F_{\sum_1^k X_i}(t)-F_{X'}(t)|$ is minimal.
\end{theorem}

\begin{proof}
From Lemma~\ref{consecutive} we get that the optimal partition for the objective function $|F_{\sum_1^k X_i}(t)-F_{X'}(t)|$ is consecutive.
Using this fact allow us to reduce our problem to the same problem solved by Rothblum in the paper from 1982. Let $X = \sum_{i=1}^k X_i$, from Rothblum we first construct the weighted graph $G = (V,E)$, where $V = \support(X), E = \{\forall i,j \in V, (i,j)\in E\}$ and the weights $\forall e=(i,j)\in E, w(e) = |F_{X}(i)-F_{X}(j)|$. In a classical Rothblum the shortest path is found by Bellman Ford algorithm, in our case we use a variation of Bellman-Ford, minimax bottleneck path by using the function $bottleneck(x) =    min   [max(bottleneck(v),w(e))]$. Bellman-Ford algorithm can be forced to return path of some length, in our case $m$, meaning, find the lightest maximal edge in a length $m$ path. 
$X'$ is the result of $OptimalTrim(X_1 \dots X_n,m)$, which is the same as described above. 
\end{proof}

\begin{algorithm}
  \DontPrintSemicolon
  \SetKwFunction{Sequence}{Sequence} 
  \SetKwFunction{Convolv}{Conv}
  \SetKwFunction{Trim}{Trim}

   $D=((0,1))$ //  Dummy random var.: $0$ with prob. $1$ \;
  \For{$i=1$ \emph{\KwTo} $k$} {
	$D$=Convolve($D, X_i$)\; 
	}
   $G=(E,V)$ \;
   $V = \support(D)$\;
   $E = \{\forall i,j \in V, (i,j)\in E\}$\;
   \ForEach{$e=(i,j) \in  E $ }{
   $w(e) = F_D(i)+\sum_{t=i}^j P_D(t)-F_D(i)$
   }
   $L, \varepsilon$=BellmanFord($G$, $min(V)$, $m$, $bottleneck(x) =    min   [max(bottleneck(v),w(e))]$) /* Run Bellman-Ford algorithm as minimax bottleneck path version, $m$ times, starting with a single source, where $L$ is the path */ \;
   %http://www.cs.cmu.edu/afs/cs/academic/class/15451-f04/www/lectures/lect1012.txt
   $D'=()$\;
   \ForEach{$e=(i,j) \in  L $ }{
   $D' = append(D', (i,\sum_{t=i}^j P_D(t))$
   }
   \Return $D, bottleneck(x)$\;
  %-------------------------------%
  \SetKwProg{myproc}{Procedure}{}{}
  \myproc{
  BellmanFord($G$,$source$,$m$,$f$)
  }{ }
     
\caption{OptimalTrim ($X_1,\dots,X_k$ , $m$)}  
\label{alg:sequence}
\end{algorithm}


Two not proved yet issues:
1) why the first gives us the minimal partition? 2) bottleneck  = the maximal error, $\varepsilon$.
\end{document}

